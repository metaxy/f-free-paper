\documentclass[12pt,a4paper,onecolumn,oneside,titlepage]{article}
\usepackage[utf8]{inputenc}
\usepackage[german]{babel}
\usepackage[T1]{fontenc}
\usepackage{amsmath}
\usepackage{amsfonts}
\usepackage{amssymb}
\author{Paul Walger}
\title{Approximation of f-free}
\makeindex
\begin{document}
\maketitle
\tableofcontents
\newpage

\section{Abstract}

\section{Einleitung}
\subsection{Definitionen}
\subsubsection{Definition Graph}
\subsubsection{Definition induzierte Subgraph}

\subsection{Motivation}
\subsection{Anwendungsbeispiele}

\section{Framework}
\subsection{Repräsentation vom Graphen}
\subsection{Das Finden von induzierten Subgraphen}

\section{Algorithmen}
\subsection{Top-Bottom}
\subsection{Bottom-Top}
\subsection{Grow-Reduce}

\section{Aufbau der Test}
\subsection{Datensätze}
\subsection{Optimale Lösung}
Um die Qualität der Lösung eines heuristischen Ansatzes bewerten zu können, ist es erforderlich die optimale Lösung zu wissen. Es gibt verschiedene Ansatz wie das Problem zu lösen sein, wir haben uns jedoch für die lineare Optimierung entschieden.
\subsubsection{Lineare Optimierung}
Bei der linearen Optimierung wird eine lineare Zielfunktion minimiert bzw. maximiert, wobei sie durch lineare Gleichungen und Ungleichungen beschränkt ist.
\subsubsection{Das Model des Graphen}

\subsubsection{Die Beschränkungen}
\subsubsection{Pseudocode}


\subsection{Formate}

\section{Auswertung}
\section{Zukünftige Forschungsmöglichkeiten}
\section{Related Work}
\section{Zusammenfassung}

\end{document}